\section*{Введение}
\label{sec:intro}

В настоящее время технология Wi-Fi стала основной технологией для беспроводных локальных сетей. Чтобы удовлетворить высокие и постоянно растущие требования к пропускной способности в беспроводных сетях Wi-Fi, комитет IEEE 802 разрабатывает IEEE 802.11ax: новое дополнение к стандарту IEEE 802.11. Оно включает в себя различные способы повышения эффективности Wi-Fi в известных сценариях. 

Большое количество устройств Wi-Fi, а также огромное количество развёрнутых плотных сетей приводят к помехам, связанными с интерференцией. Стандарт рассматривает такие сценарии, где много сетей находятся рядом друг с другом и требуется снизить их влияние друг на друга. Другая часть решений направлена на улучшение процесса передачи в случае одиночной точки доступа. 

Главной особенностью стандарта IEEE 802.11ax и одним из способов улучшить эффективность сетей Wi-Fi является применение технологии  поддержки множественного доступа с ортогональным частотным разделением OFDMA, которая расширяет стандартный CSMA/CA путём введения возможности разделения ресурсов канала в частотной области. 
Начиная со стандарта IEEE 802.11a, устройства Wi-Fi используют технологию OFDM.
Однако с OFDM все поднесущие используются для передачи данных для всего лишь одного пользователя в один момент времени, тогда как OFDMA позволяет назначать различные тоны для использования различными пользователями в один и тот же момент времени. 
С другой стороны, эффективность применения OFDMA значительно зависит от того, как тоны распределяются между пользователями. 
При этом стандарт 802.11ax предоставляет всего лишь гибкий набор инструментов для распределения ресурсов без каких-либо предопределённых алгоритмов планирования.
 
Стоит отметить, что технология OFDMA существовала и ранее: она используется в сотовых сетях, которые работают в лицензированном спектре, в частности "--- в сетях LTE. Для технологии Wi-Fi OFDMA является совершенно новым механизмом. 

Упомянутая проблема планирования ресурсов между пользователями была тщательно исследована в сотовых сетях LTE. 
Таким образом, на первый взгляд, стоит использовать один из существующих для сетей LTE планировщиков и адаптировать его к особенностям применения OFDMA в Wi-Fi. 
Поскольку фундаментальные принципы работы OFDMA в Wi-Fi отличаются от принципов ее работы в сетях LTE, очевидно, что этот процесс адаптации сам по себе непрост. Более того, предположения, используемые при разработке планировщиков ресурсов в LTE, и вовсе становятся неверными относительно сетей 802.11ax. 
Как следствие, никто не может гарантировать, что лучший планировщик LTE останется таким же хорошим после его адаптации к сетям 802.11ax. 

В данной работе рассматривается проблема планирования ресурсов в сетях IEEE 802.11ax с использованием OFDMA. Также ставится цель уменьшить среднее время передачи данных, для достижения которой разрабатывается новый планировщик ресурсов. Показывается, что этот планировщик работает лучше известных аналогов для сетей 802.11ax.

Дальнейшее изложение устроено следующим образом. В разделе~\ref{sec:scenario} дано подробное описание объекта исследования и постановка задачи.
В разделе~\ref{sec:litra} приведён обзор литературы.
Раздел~\ref{sec:scheduler} описывает планировщик, разработанный для уменьшения времени доставки в сетях 802.11ax. В конце главы даны численные результаты. В заключении приведены основные выводы, которые можно сделать в результате исследования гибридного доступа при передаче данных в сетях IEEE 802.11ax.