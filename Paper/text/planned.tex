\section{Исследование планирования ресурсов в OFDMA}
\label{sec:scheduler}

\subsection{Предпосылки и описание известного планировщика ресурсов}
Проблема распределения ресурсов в сотовых сетях широко изучена.
Во многих работах она сформулирована как некоторая задача оптимизации. 
Рассмотрим сеть с базовой станцией (БС) и $n$ ассоциированных с ней пользователями, у которых есть данные на передачу. 
Периодически базовая станция запускает планировщик, который распределяет $m$ ресурсных блоков между пользователями таким образом, чтобы максимизировать некоторую функцию полезности, зависящую от параметров клиентских станций и сети. 

Из литературы известен планировщик {\foreignlanguage{english}{(SRTF)Shortest Remaining Time First}}, целью которого является обеспечивать минимальное среднее время загрузки, если свойства канала не изменяются со временем, и скорость в разных РБ одинакова и аддитивна. Причём данный планировщик не использует возможность частотного разделения канала и всегда выдаёт весь канал какой-то станции. 

За исключением ожидания, время, необходимое для передачи потока $i$, равно $t_i = \frac{D_i}{r_{i}}$, где $D_i(t)$ "--- это оставшийся объем данных пользователя  $i$, $r_i$ "--- его скорость во всем канале. Первая станция заканчивает доставку потока к моменту времени $t_1$. Вторая станция начинает свою передачу сразу после первой и доставляет свой поток по времени $t_1 + t_2$ и т. д. В результате общее время передачи для существующих потоков равняется 
$$T =t_1 +(t_1+t_2) + (t_1+t_2+t_3) +\dots  =\sum_{i = 1}^{n} \left(n - i + 1\right) t_i.$$
Очевидно, чтобы минимизировать суммарное время передачи, нужно сортировать станции в порядке возрастания по $t_i$. Таким образом, SRTF распределяет все РБ пользователю, для которого $\frac{D_i(t)}{r_i}$ минимальный.  То есть, планировщик расставляет станции по возрастанию этого параметра, как показано на рис.~\ref{fig:SRTF}.
\begin{figure}[tb]
	\centering
	\begin{scaletikzpicturetowidth}{0.8\textwidth}
		\begin{tikzpicture}[scale = \tikzscale]
				\draw[draw = none, pattern= horizontal lines light gray, pattern color=black, ] (1.4,1) rectangle (3,3);
				
				\draw[draw = none, pattern=checkerboard light gray, pattern color=black, ] (3,1) rectangle (5,3);
				
				\draw[draw = none, pattern=dots, pattern color=black, ] (5,1) rectangle (8.5,3);
		
		\draw (1.4,3) -- (1.4,1);
		\draw (3,3) -- (3,0.75);
		\draw (5,3) -- (5,0.75);
		\draw (8.5,3) -- (8.5,0.75);
		
		\draw [arrows={-triangle 45}] (-0.5,1) -- (12,1);
		
		\draw [arrows={-triangle 45}] (0.2,1) -- (0.2,3.7);
		
		\node at (12,  1.2) {\textit{время}};
		\node at (1.25,  3.5) {\textit{частота}};
		
		\node at (2.2,2) {\textbf{STA 1}};
		\node at (4,2) {\textbf{STA 2}};
		\node at (6.7,2) {\textbf{STA 3}};
		
		\scriptsize
		\node at (3,  0.55) {$t_1$};
		\node at (5,  0.55) {$t_1+t_2$};
		\node at (8.5,  0.55) {$t_1+t_2 + t_3$};
		
		\end{tikzpicture}
	\end{scaletikzpicturetowidth}
	\caption{\label{fig:SRTF} Работа планировщика SRTF}
\end{figure}

\subsection{Описание разработанного планировщика}

В этом разделе описывается новый планировщик MUTAX (Minimizing Upload Time in 802.11ax networks), направленный на то, чтобы уменьшить время передачи в сетях 802.11ax. Для этого модифицируем планировщик SRTF, чтобы получить новый, более эффективный планировщик для сетей IEEE 802.11ax.
Здесь пренебрегаем эффектами, связанными с различными накладными расходами (включая агрегацию и фрагментацию). 
Кроме того, для краткости мы рассматриваем только $n$ станций с потоками данных и предполагаем, что каждая станция имеет только один поток. 
Пусть станция (как и соответствующий ей поток) обозначены номером $i$, $i = 1,\dots,n$. 

Назовем слотом временной интервал между двумя последующими триггер-кадрами.
Следует отметить, что слоты могут иметь разную продолжительность. 
Максимальная продолжительность связана со стандартным пределом 5484 мкс для продолжительности блока данных физического протокола (PPDU). 
Длительность слота определяется по самой длительной передаче.

Алгоритм MUTAX имеет два этапа работы. 
На первом этапе вычисляется время доставки всех известных потоков так, как если бы использовалось исчерпывающее обслуживание. 
На втором этапе алгоритм пытается улучшить это суммарное время передачи, параллельно используя несколько потоков.

Давайте подробно рассмотрим шаги. На первом этапе проводится расчёт с предположением, что каждый поток обслуживается исчерпывающим образом, т.е. подобно алгоритму планировщика SRTF, в котором точка доступа предоставляет весь канал одной станции для передачи данных. Кроме ожидания, время, необходимое для передачи потока данных станции $i$, равно $t_i = \frac{D_i}{r_{i}}$, где $r_i$ "--- скорость станции $i$, передающей во всей частотной полосе. В таком случае первая станция заканчивает доставку потока к моменту времени $t_1$. Вторая станция начинает свою передачу сразу после первой и доставляет свой поток ко времени $t_2 + t_1$ и т. д. В результате общее время передачи для существующих потоков равняется
\[ T_{step1} = \sum_{i = 1}^{n} \left(n - i + 1\right) t_i. \]
Очевидно, чтобы минимизировать суммарное время передачи, нужно сортировать станции в порядке возрастания по $t_i$, что и делает на первом этапе планировщик MUTAX.

На втором этапе алгоритм предполагает деление канала на несколько РБ в текущем слоте, как показано на рис.~\ref{fig:mutexblat}. Заметим, что на рисунке приведен пример конфигурации, когда некоторая часть ресурсов отведена под случайный доступ. Итак, пусть $m$ "--- количество РБ и $j$, $1\leq j \leq m$ "--- порядковый номер РБ в рассмотренной конфигурации ресурсов. 
Обозначим $x_i^j$ за индикаторную величину, которая равняется $1$, если станция $i$ ассоциирована с блоком $j$, и $0$, в противном случае.
Также пусть $X$ "--- это двумерная матрица $\left\{x_i^j\right\}$, описывающая связь между РБ и станциями. 
\begin{figure}[htb]
\centering
\begin{scaletikzpicturetowidth}{\textwidth}
\begin{tikzpicture}[scale = \tikzscale]
\draw[draw = none, pattern=north east lines, pattern color=black, ] (1.7,2) rectangle (2.6,3);
\draw[draw = none, pattern=checkerboard light gray, pattern color=black, ] (1.7,1.56666) rectangle (2.6,2);
\draw[draw = none, pattern=dots, pattern color=black, ] (1.7,1.12222) rectangle (2.6,1.56666);

\draw[draw = none, pattern=north east lines, pattern color=black, ] (2.6,1) rectangle (3.4,3);
\node at (3,2) {\textbf{1}};
\draw[draw = none, pattern=checkerboard light gray, pattern color=black, ] (3.4,1) rectangle (5.2,3);
\node at (4.25,2) {\textbf{STA 2}};
\draw[draw = none, pattern=dots, pattern color=black, ] (5.2,1) rectangle (8,3);


\draw [arrows={-triangle 45}] (-0.5,1) -- (12,1);

\draw [arrows={-triangle 45}] (0.2,1) -- (0.2,3.7);


\draw (2.6,3) -- (2.6,0.75);
\draw (3.4,3) -- (3.4,0.75);
\draw (5.2,3) -- (5.2,0.75);
\draw (8,3) -- (8,0.75);

\node at (2.15,2.5) {\textbf{1}};
\node at (2.15,1.8) {\textbf{2}};
\node at (2.15,1.3) {\textbf{3}};

\node at (12,  1.2) {\textit{время}};
\node at (1.25,  3.5) {\textit{частота}};

\node at (6.5,2) {\textbf{STA 3}};

\scriptsize
\node at (2.6,0.5) {\vphantom{$\tilde t_1$} $\tau$};
\node at (3.4,0.5) {$\tau+\tilde t_1$};
\node at (5.2,0.5) {$\tau+\tilde t_1 + \tilde t_2$};
\node at (8,0.5) {$\tau+ \tilde t_1 + \tilde t_2+ \tilde t_3$};
\end{tikzpicture}
\end{scaletikzpicturetowidth}
\caption{\label{fig:mutexblat} К планировщику MUTAX}
\end{figure}

При введенных обозначениях общее время загрузки $T\left(X\right)$ существующих потоков отличается от $T_{step1}$ следующим образом. Во-первых, время передачи каждого из  $n$ потоков увеличивается на длительность текущего временного слота $\tau$. 
Во-вторых, если $x_i^j=1$, оставшийся объем данных потока  уменьшается на размер данных, которые станция передает в РБ $j$ текущего слота: $\Delta D_i^j = \min\left\{D_i, \tau \times r_{i}^{j}\right\}$.
Таким образом,
\[ T\left(X\right) = n \tau + \sum_{i = 1}^{n} \left(n - i + 1\right) \frac{D_i -  \sum_{j = 1}^{m} x_i^j \Delta D_i^j}{r_{i}} \]

Поскольку и $\tau$, и  $\Delta D_i^j$ зависят от распределения ресурсов $X$, минимизация $T(X)$ требует поиска по возможным способам распределения РБ для станций, и для упрощения задачи здесь предлагается эвристический подход, основанный на двух предположениях. 
Во-первых, допустимо пренебрегать изменением $n \tau$ для разных распределений, поскольку слот не может быть слишком длинным из-за стандартных ограничений. 
Во-вторых, сортируем станции в порядке возрастания $t_n$ только один раз, прежде чем рассматривать разные способы назначения РБ станциям. 
В этих предположениях, чтобы минимизировать $T(X)$, нужно максимизировать следующее выражение 
$$\sum_{j = 1}^{m} \sum_{i = 1}^{n } \left(n - i + 1\right) \frac{\Delta D_i^j}{r_{i}}.$$

Определим параметр $\lambda_i^j = \left(n - i\right) \frac{\Delta D_i^j}{r_{i}}$, и тогда задача планирования ресурсов сводится к следующей оптимизационной задаче:
\begin{align} 
\max \sum_{i} \sum_{j} x_i^j \lambda_i^j; \\ \label{usl312}
\text{при} \sum_{i} x_i^j \leq 1,\ \  \forall j; \\ \label{usl313}
\sum_{j} x_i^j \leq 1, \ \ \forall i; \\ \label{usl314}
\sum_{i} \sum_{j} x_i^j \leq m,
\end{align}
в которой условие \eqref{usl312} олицетворяет тот факт, что одной станции не может быть назначено больше, чем один ресурсный блок согласно ограничениям применения технологии OFDMA; условие \eqref{usl313}: одному ресурсному блоку не может быть поставлено в соответствие больше одной станции; последнее условие \eqref{usl314} "--- суммарно всего имеем $m$ ресурсных блоков.

Эта задача известна как проблема назначения, которая может быть решена в полиномиальное время с использованием венгерского алгоритма \cite{bourgeois1971extension}, известного также как алгоритм Куна"--~Манкреса или алгоритм Манкреса решения задачи о назначениях. 
Его решением является назначение $\hat X$, дающее время $T(\hat X)$. 

Следует заметить, что такая задача оптимизации "--- это общая математическая задача. Например, если взять за параметр $\lambda_i^j$ величину, равную $r_i^j / S_i$, то, получится адаптация планировщика PF к сетям 802.11ax.

Отметим, что назначение найдено для конкретной конфигурации РБ. Чтобы в конце концов минимизировать время загрузки, необходимо рассматривать различные способы разделения канала на РБ.
Поскольку эффективность алгоритма планировщика MUTAX значительно зависит от того, как канал разделен на РБ, то для каждой конфигурации РБ решается задача оптимизации и находится $T(X)$ для каждой известной $X$
и, таким образом, найдя минимальное время среди всех полученных результатов находим наилучшую конфигурацию ресурсов канала и их лучшее назначение среди всех доступных. 
Перебор конфигураций РБ может быть ускорен за счёт исключения конфигураций, которые явно хуже известных альтернатив, например, если у нас есть две станции и канал 20 МГц, рассматривая конфигурацию с одним 106-тоновым РБ, одним 52-тоновым РБ и тремя 26-тоновые, можно исключить конфигурацию с одним 106-тоновым РБ и пятью 26-тоновыми~РБ.

\iffalse
Планировщик MaxRate в качестве целевой функции рассматривает суммарную пропускную способность в канале. Данный планировщик рассматривает ресурсные блоки один за другим и присваивает каждому РБ пользователя с наивысшей номинальной скоростью передачи данных $r_{i}^{j}$ в этом РБ.

Планировщик Proportional Fair максимизирует сумму логарифмов пропускной способности станций в заданный момент времени, или, другими словами, их среднее геометрическое. Планирование ресурсов сводится к решению оптимизационной задачи, аналогичной MUTAX, но вместо параметра $\lambda$ брать отношение скорости передачи в ресурсном блоке к количеству переданных данных. Поскольку производительность алгоритма планировщика PF значительно зависит от того, как канал разделен на РБ, то и для его работы находим лучшую конфигурацию РБ перебором.
\fi