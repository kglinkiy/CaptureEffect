\subsection{Численные результаты}

Имитационное моделирование проводилось в следующем сценарии. 
Сеть работает в канале 40~МГц на частоте 5~ГГц. 
Размеры потока берутся из усеченного логнормального распределения с минимальными, средними и максимальными значениями 1~КБ, 500~КБ и 5~МБ соответственно. 
Когда поток доставляется, следующий поток генерируется после случайной задержки, полученной из усеченного экспоненциального распределения с минимальными, средними и максимальными значениями 0.1~с, 0.3~с и 0.6~с соответственно.
\iffalse
\begin{figure}%
	\centering
	\subfloat[label 1]{{\includegraphics[width=5cm]{5-z.jpg} }}%
	\qquad
	\subfloat[label 2]{{\includegraphics[width=5cm]{20-z.jpg} }}%
	\caption{2 Figures side by side}%
	\label{fig:example}%
\end{figure}
\fi
\begin{figure}[thb]
\centering
\includegraphics[width=0.6\textwidth]{5-z.jpg}
\caption{Зависимость времени доставки от количества станций c применением разных планировщиков, когда станции находятся на расстоянии 5~метров}\label{fig:10-e}
\end{figure}
В первом эксперименте станции расположены равномерно в пределах небольшого круга радиуса $R = 5$~м вокруг точки доступа. 
В таком случае качество канала настолько хорошее, что станции используют максимально доступную СКК для передачи в РБ любой ширины.
Очевидно, что в этом случае OFDMA не может принести никакой прибыли против SRTF, поскольку разделение канала без изменения СКК не может увеличить скорость передачи данных при условии, что СКК может справиться с шумом. Этот результат поддерживается результатами моделирования, см. рис.~\ref{fig:10-e}, который показывает, что если канал для всех станций является идеальным, планировщик MUTAX даёт одинаковое время загрузки, как SRTF, и они оба превосходят аналоги планировщиков PF и MR, адаптированных под сети IEEE 802.11ax, на 30\%. 

Второй эксперимент соответствует случаю, когда станции расположены в большем круге радиуса $R = 20$~м. Такие условия обеспечивают разнообразие выбора СКК, поэтому становится целесообразным разделить канал между разными пользователями.
Согласно результатам моделирования, см. рис.~\ref{fig:25-e}, в этом случае для минимизации времени передачи классический SRTF работает намного хуже, чем даже адаптация PF для сетей 802.11ax. В то же время MUTAX показывает на 20\% меньшее время передачи, чем PF. 
\begin{figure}[tbh]
	\centering
	\includegraphics[width=0.6\textwidth]{20-z.jpg}
	\caption{Зависимость времени доставки от количества станций c применением разных планировщиков, когда станции находятся на расстоянии 20 метров}\label{fig:25-e}
\end{figure}

Результаты показывают, что в этом случае планировщики, основанные на исчерпывающем обслуживании (MR и SRTF), намного менее эффективны, чем планировщики разбиения каналов (MUTAX и PF), а разрыв между ними увеличивается с количеством клиентских станций.
В сценарии большого круга выигрыш в MUTAX по сравнению с SRTF и MR составляет почти 100\%.

