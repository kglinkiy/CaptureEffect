\section{Обзор литературы}
\label{sec:litra}

Несмотря на то, что комитет IEEE 802 планирует опубликовать стандарт IEEE 802.11ax только в 2019 году, он уже широко исследован в литературе~\cite{khorov2016several, ofdma-par1, ofdma-par2, karaca2016resource, lanante2017performance}. 

Авторы \cite{khorov2016several} изучают производительность сети, состоящей как из станций, поддерживающих стандарт современного Wi-Fi, так и станций, которые могут работать согласно стандарту IEEE 802.11ax. 
Авторы предлагают подход для оптимального выбора значений параметров канала, которые гарантируют справедливость в распределении ресурсов между современными станциями и станциями 802.11ax, и значительно увеличивает количество OFDMA-передач для последних. 
Однако модель, описанная в \cite{khorov2016several}, позволяет только оценивать количество OFDMA-передач в сети, но не возможно достигаемую скорость передачи данных. 

Многие исследования производительности сетей 802.11ax были представлены на совещаниях групп IEEE 802.11ax, например, \cite{ofdma-par1, ofdma-par2}. 
Однако, хотя многие сетевые топологии уже были изучены, единственным рассмотренным планировщиком был случайный планировщик со статически зафиксированной конфигурацией ресурсных блоков.
 
Проблема планирования ресурсов в восходящих потоках данных в сетях 802.11ax рассмотрена в  \cite{karaca2016resource}, в которой основное внимание уделяется выбору длительности ресурсных блоков. 
Авторы изначально предполагают, что точка доступа не знает количество данных, которые станции имеют на передачу, и, следовательно, не знает, какой длительности ресурсные блоки должны быть выделены. 
Авторы предлагают схему, которая может использоваться точкой доступа для получения этой информации и описывают лучший способ выбрать продолжительность передачи станций с точки зрения пропускной способности и потребления энергии.
Однако они не рассматривают какой-либо конкретный способ разделить канал на ресурсные блоки и назначить блоки станциям. 
Они также не рассматривают случайный доступ и возможность использования станциями для доставки запросов на передачу данных путём агрегирования их с передаваемыми данными.

Случайный доступ для восходящих потоков данных OFDMA в 802.11ax изучен в \cite{lanante2017performance}. 
Авторы рассматривают сценарий, в котором станции передают насыщенные потоки данных и используют только случайный доступ для передачи. 
Они описывают математическую модель передачи с применением метода случайного доступа OFDMA и используют её для анализа и оптимизации производительности сети с точки зрения пропускной способности и процента успешных РБ.
Методология, предложенная в статье, может быть использована для настройки случайного доступа для восходящих потоков данных OFDMA, однако использование детерминированного доступа более целесообразно с точки зрения эффективности канала для изучаемого сценария. Следует также отметить, что передача насыщенных потоков данных в восходящем потоке не является типичным сценарием для пользовательских сетей Wi-Fi, т.к. обычно трафик в восходящем потоке состоит из коротких и редко передаваемых пакетов, например, HTTP-запросов. 

В литературе ещё не было исследовано совместное использование случайного и детерминированного доступов в сетях Wi-Fi IEEE 802.11ax, что указывает на актуальность поставленной в работе задачи. 

