\section{Сценарий и постановка задачи}
\label{sec:scenario}

\subsection{Особенности применения технологии OFDMA в Wi-Fi}

У технологии OFDMA есть ряд принципиальных особенностей, которые ограничивают возможности планировщика ресурсов в Wi-Fi.

Кадры OFDMA начинаются с общей преамбулы, которую можно декодировать устройствами даже предыдущих поколений. 
Получив преамбулу, станция узнает длительность кадра и после считает среду занятой до конца длительности кадра. Оставшаяся часть кадра может быть декодирована только устройствами, поддерживающими стандарт 802.11ax. 
Эта часть может быть сформирована в соответствии с концепцией OFDMA, то есть различные группы тонов здесь могут быть назначены различным станциям и использоваться для передачи информации, предназначенной этим различным станциям.

Канал может быть разбит в наборы OFDMA-поднесущих (тонов). 
В сетях 802.11ax группа тонов, отведённых одной станции, называется ресурсным блоком (РБ). Стандарт IEEE 802.11ax определяет ресурсные блоки размером в 26, 52, 106, 242, 484, 996 и 2 $\times$ 996 OFDM тонов.
Набор доступных РБ зависит от ширины канала: так, например, в канале 40~МГц, станция может использовать РБ размером до 484 тонов. 

Существуют правила деления ресурсных блоков на более мелкие по размеру. 
Общая тенденция: широкие РБ могут быть разделены на примерно вдвое более узкие РБ независимо от других.
Блоки размером 52, 106, 484 могут быть разбиты на два ресурсных блока, следующих по уменьшению размера, независимо от других РБ. 
Как исключения, в то же время 996 и 242-тоновые РБ могут быть разбиты минимум в три РБ: два примерно в 2 раза \'{у}же размером и один 26-тоновый РБ, как показано на рис.~\ref{fig:resource_units}. 
Т.е. 996-тоновый РБ можно разделить на два 484-тоновых РБ и один 26-тоновый РБ. А 242-тоновый РБ можно разделить на два 106-тоновых РБ и один 26-тоновый РБ, причем их относительное расположение в частотной полосе строго закреплено. Данное правило деления канала возникает из-за необходимости иметь для каждого ресурсного блока служебные тоны, находящиеся с ним в непосредственной близости в частотной области. 
\begin{figure}[!h]
	\centering
    \begin{tikzpicture}[scale = 1,0]
    \draw [line width=0.2mm] (0.00, 0.00) rectangle (9.00, 0.80);
    \node [text width=1.5cm, align=center] at (4.0,  0.4) {484};
    \node [text width=1.5cm, align=center] at (5.0,  0.4) {тонов};
    
    \draw [line width=0.2mm] (0.00, 1.0) rectangle (4.50, 1.8);
    \draw [line width=0.2mm] (4.50, 1.0) rectangle (9.00, 1.8);
    \node [text width=1.5cm, align=center] at (2.5,  1.4) {242};
    \node [text width=1.5cm, align=center] at (7.0,  1.4) {242};
    
    \draw [line width=0.2mm] (0.00, 2.0) rectangle (2.00, 2.8);
    \draw [line width=0.2mm] (2.00, 2.0) rectangle (2.50, 2.8);
    \draw [line width=0.2mm] (2.50, 2.0) rectangle (4.50, 2.8);
    \draw [line width=0.2mm] (4.50, 2.0) rectangle (6.50, 2.8);
    \draw [line width=0.2mm] (6.50, 2.0) rectangle (7.00, 2.8);
    \draw [line width=0.2mm] (7.00, 2.0) rectangle (9.00, 2.8);
    \node [text width=1.5cm, align=center] at (1.0,  2.4) {106};
    \node [text width=1.5cm, align=center] at (2.25, 2.4) {26};
    \node [text width=1.5cm, align=center] at (3.5,  2.4) {106};
    \node [text width=1.5cm, align=center] at (5.5,  2.4) {106};
    \node [text width=1.5cm, align=center] at (6.75, 2.4) {26};
    \node [text width=1.5cm, align=center] at (8.0,  2.4) {106};
    
    \draw [line width=0.2mm] (0.00, 3.0) rectangle (1.00, 3.8);
    \draw [line width=0.2mm] (1.00, 3.0) rectangle (2.00, 3.8);
    \draw [line width=0.2mm] (2.00, 3.0) rectangle (2.50, 3.8);
    \draw [line width=0.2mm] (2.50, 3.0) rectangle (3.50, 3.8);
    \draw [line width=0.2mm] (3.50, 3.0) rectangle (4.50, 3.8);
    \draw [line width=0.2mm] (4.50, 3.0) rectangle (5.50, 3.8);
    \draw [line width=0.2mm] (5.50, 3.0) rectangle (6.50, 3.8);
    \draw [line width=0.2mm] (6.50, 3.0) rectangle (7.00, 3.8);
    \draw [line width=0.2mm] (7.00, 3.0) rectangle (8.00, 3.8);
    \draw [line width=0.2mm] (8.00, 3.0) rectangle (9.00, 3.8);
    \node [text width=1.5cm, align=center] at (0.5,  3.4) {52};
    \node [text width=1.5cm, align=center] at (1.5,  3.4) {52};
    \node [text width=1.5cm, align=center] at (2.25, 3.4) {26};
    \node [text width=1.5cm, align=center] at (3.0,  3.4) {52};
    \node [text width=1.5cm, align=center] at (4.0,  3.4) {52};
    \node [text width=1.5cm, align=center] at (5.0,  3.4) {52};
    \node [text width=1.5cm, align=center] at (6.0,  3.4) {52};
    \node [text width=1.5cm, align=center] at (6.75, 3.4) {26};
    \node [text width=1.5cm, align=center] at (7.5,  3.4) {52};
    \node [text width=1.5cm, align=center] at (8.5,  3.4) {52};
    
    \draw [line width=0.2mm] (0.00, 4.0) rectangle (0.50, 4.8);
    \draw [line width=0.2mm] (0.50, 4.0) rectangle (1.00, 4.8);
    \draw [line width=0.2mm] (1.00, 4.0) rectangle (1.50, 4.8);
    \draw [line width=0.2mm] (1.50, 4.0) rectangle (2.00, 4.8);
    \draw [line width=0.2mm] (2.00, 4.0) rectangle (2.50, 4.8);
    \draw [line width=0.2mm] (2.50, 4.0) rectangle (3.00, 4.8);
    \draw [line width=0.2mm] (3.00, 4.0) rectangle (3.50, 4.8);
    \draw [line width=0.2mm] (3.50, 4.0) rectangle (4.00, 4.8);
    \draw [line width=0.2mm] (4.00, 4.0) rectangle (4.50, 4.8);
    \draw [line width=0.2mm] (4.50, 4.0) rectangle (5.00, 4.8);
    \draw [line width=0.2mm] (5.00, 4.0) rectangle (5.50, 4.8);
    \draw [line width=0.2mm] (5.50, 4.0) rectangle (6.00, 4.8);
    \draw [line width=0.2mm] (6.00, 4.0) rectangle (6.50, 4.8);
    \draw [line width=0.2mm] (6.50, 4.0) rectangle (7.00, 4.8);
    \draw [line width=0.2mm] (7.00, 4.0) rectangle (7.50, 4.8);
    \draw [line width=0.2mm] (7.50, 4.0) rectangle (8.00, 4.8);
    \draw [line width=0.2mm] (8.00, 4.0) rectangle (8.50, 4.8);
    \draw [line width=0.2mm] (8.50, 4.0) rectangle (9.00, 4.8);
    \node [text width=1.5cm, align=center] at (0.25, 4.4) {26};
    \node [text width=1.5cm, align=center] at (0.75, 4.4) {26};
    \node [text width=1.5cm, align=center] at (1.25, 4.4) {26};
    \node [text width=1.5cm, align=center] at (1.75, 4.4) {26};
    \node [text width=1.5cm, align=center] at (2.25, 4.4) {26};
    \node [text width=1.5cm, align=center] at (2.75, 4.4) {26};
    \node [text width=1.5cm, align=center] at (3.25, 4.4) {26};
    \node [text width=1.5cm, align=center] at (3.75, 4.4) {26};
    \node [text width=1.5cm, align=center] at (4.25, 4.4) {26};
    \node [text width=1.5cm, align=center] at (4.75, 4.4) {26};
    \node [text width=1.5cm, align=center] at (5.25, 4.4) {26};
    \node [text width=1.5cm, align=center] at (5.75, 4.4) {26};
    \node [text width=1.5cm, align=center] at (6.25, 4.4) {26};
    \node [text width=1.5cm, align=center] at (6.75, 4.4) {26};
    \node [text width=1.5cm, align=center] at (7.25, 4.4) {26};
    \node [text width=1.5cm, align=center] at (7.75, 4.4) {26};
    \node [text width=1.5cm, align=center] at (8.25, 4.4) {26};
    \node [text width=1.5cm, align=center] at (8.75, 4.4) {26};  
    \end{tikzpicture}
    \end{scaletikzpicturetowidth}
	\caption{\label{fig:resource_units} Расположение РБ в 40 МГц канале.}
\end{figure}

В сетях LTE планировщик может выделять произвольное подмножество РБ, предназначенных для нисходящих потоков данных, на пользователя или произвольный интервал РБ в восходящих потоках на пользователя. 
802.11ax накладывает гораздо более строгие ограничения на распределение РБ, поскольку как в нисходящих потоках, так и в восходящих, одной станции не может быть назначено более одного ресурсного блока. 
При этом размер этого ресурсного блока может варьироваться указанным выше способом. 
Другими словами, например, для полосы в 40~МГц (здесь всей полосе соответствуют 484 тона) имеется всего 18 26-тоновых ресурсных блоков. В дальнейшем размер ресурсного блока измеряется в размерах самого маленького ресурсного блока (26 тонового). Иначе говоря, в случае канала в 40~МГц планировщик может разделять ресурсы канала в группы размера 1, 2, 4, 9, 18 таким образом, чтобы суммарный размер остался меньше или равен 18. Согласно стандарту, данное ограничение о специфичном разбиении на ресурсные блоки появляется из-за надобности в служебных тонах для передачи в нелицензированном спектре, которые должны располагаться в непосредственной близости у каждого выделяемого РБ в частотной полосе.

Также стандарт запрещает использование низкоскоростных сигнально кодовых конструкций (СКК) в РБ малого размера.
Размер ресурсного блока определяет набор СКК, которые могут быть использованы для передачи в РБ. 
В частности, новый QAM-1024 может использоваться только в 242-тоновых и б\'{о}льших РБ.  
Следует отметить, что передача в более широком РБ не всегда означает, что станция будет передавать на большей скорости. 
Станция использует одну и ту же мощность как для широких, так и для узких РБ. 
Отсюда следует ожидать б\'{о}льшее значение отношения сигнал-шум в узких РБ. А значит, в узких РБ станции могут использовать более скоростные СКК. 
Такие ограничения затрудняют задачу планирования ресурсов в сетях 802.11ax. 

Конфигурации ресурсных блоков могут содержать блоки с различным количеством тонов, однако все РБ внутри кадра должны иметь одинаковую продолжительность. Для этого станции могут использовать новую гибкую фрагментацию, предусмотренную в стандарте.

Все передачи внутри одного кадра OFDMA должны быть синхронизированы, то есть начинаться и заканчиваться одновременно. Это можно легко реализовать в нисходящих потоках данных, где кадр OFDMA генерируется точкой доступа. Однако для обеспечения синхронизации для восходящих потоков данных точка доступа может использовать триггер-кадр (ТК). Синхронизация проводится следующим образом. Точка доступа отсылает ТК пользовательским станциям, после этого, через SIFS после приёма ТК, станции передают свои части кадра с использованием OFDMA. При необходимости точка доступа подтверждает приём каждой части, отправив набор кадров подтверждений внутри кадра OFDMA или отправив блочное подтверждение (БП) нескольким станциям через SIFS после кадра станций.

В 802.11ax точка доступа определяет СКК, длительность передачи, распределение РБ и другие параметры OFDMA. Такая информация может передаваться в заголовках кадров для нисходящих и в триггер-кадре для восходящих потоков. Если необходимо, то точка доступа подтверждает приём данных от каждой клиентской станции, передавая набор кадров-под\-тверж\-де\-ний в многопользовательском блочном подтверждении. Пример передачи в восходящих потоках с помощью OFDMA показан на рис.~\ref{fig:transmission}. 
\begin{figure}[thb]
	\centering
		\begin{scaletikzpicturetowidth}{0.75\textwidth}
		\begin{tikzpicture}[scale = \tikzscale]
        \footnotesize
        \draw [arrows={-triangle 45}] (0,0.8) -- (7.5,0.8);
        \draw [arrows={-triangle 45}] (0,0.8) -- (0,3.2);
        \node at (7.3,  0.5) {\textit{Время}};
        \node at (0.7,  3.1) {\textit{Частота}};
        \draw [line width=0.5mm] (1, 0.8) rectangle (2, 2.6);
        \node [text width=1.5cm, align=center] at (1.5,  1.8) {ТК};
        \draw [line width=0.5mm] (3, 0.8) rectangle (5, 1.3);
        \draw [line width=0.5mm] (3, 1.3) rectangle (5, 1.8);
        \draw [line width=0.5mm] (3, 1.8) rectangle (5, 2.1);
        \draw [line width=0.5mm] (3, 2.1) rectangle (5, 2.6);
        \node [text width=2cm, align=left] at (4, 1.05) {Станция 1};
        \node [text width=2cm, align=left] at (4, 1.55) {Станция 2};
        \node [text width=2cm, align=left] at (4.3, 1.95) {\quad...};
        \node [text width=2cm, align=left] at (4, 2.35) {Станция $x$};
        \draw [line width=0.5mm] (6, 0.8) rectangle (7.2, 2.6);
        \node [text width=1.5cm, align=center] at (6.6,  1.8) {БП};
        \draw (2.0,  0.3) -- (2.0,  0.8);
        \draw (3.0,  0.3) -- (3.0,  0.8);
        \draw [arrows={triangle 45-triangle 45}] (2.0,0.3) -- (3.0,0.3);
        \node at (2.5,  0.5) {\small$SIFS$};
        \draw (5.0,  0.3) -- (5.0,  0.8);
        \draw (6.0,  0.3) -- (6.0,  0.8);
        \draw [arrows={triangle 45-triangle 45}] (5.0,0.3) -- (6.0,0.3);
        \node at (5.5,  0.5) {\small$SIFS$};
        \end{tikzpicture}
        \end{scaletikzpicturetowidth}
\caption{\label{fig:transmission} Восходящие потоки данных в OFDMA.}
\end{figure}

Как видно, OFDMA имеет много достоинств. 
Во-первых, передача становится более надёжной при частотно-избирательных замираниях и затуханиях. 
Это особенно важно для широких каналов с частотой 160 МГц, введённых в дополнении IEEE 802.11ac. 
Во-вторых, с помощью OFDMA возможна <<склейка>> коротких пакетов, предназначенных разным станциями или передаваемых разными станциями, что значительно снижает накладные расходы, вызванные заголовками физического уровня технологии Wi-Fi, и тем самым снижает время доступа к каналу.
В-третьих, для передачи восходящих потоков с помощью станций, находящихся на пороге чувствительности мощности, имеет смысл использовать узкие каналы вместо широких. 
Действительно, поскольку станции пространственно разделены, они могут одновременно передавать с максимальной допустимой мощностью без нарушения законных ограничений на излучаемую энергию.
Таким образом, уменьшая ширину РБ, мы увеличиваем спектральную плотность мощности, получаемую от станции, и можем использовать более высокие СКК. 
Другими словами, увеличивается принятая кумулятивная спектральная мощность по сравнению со случаем, когда передаёт всего лишь одна станция. 
Для этих станций увеличение спектральной плотности мощности приводит к более высокоскоростной СКК, поэтому средний объём данных, полученных из кадра OFDMA, выше, чем у ранее использованных технологий. 
Таким образом, в отличие от LTE в сетях 802.11ax скорости в ресурсных блоках, используемых для восходящих потоков, не являются аддитивными, то есть если станция передаёт в вдвое большем РБ, не гарантируется, что она передаст вдвое больше данных. Возможны даже сценарии, когда она передаст меньшее количество информации в подобном случае, что подчеркивает необходимость разработки нового планировщика для сетей 802.11ax.
